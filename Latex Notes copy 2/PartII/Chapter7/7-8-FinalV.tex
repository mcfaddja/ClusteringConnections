\documentclass[../ClusteringConnectionsMAIN.tex]{subfiles}
 



\begin{document}
\begin{flushleft}
\begin{large}

To construct the sets of $\mathbb{V}^\star$, we look for unique $\mathbb{V}_i \in \mathbb{V}$ and then store each unique $\mathbb{V}_i$ as its own set in $\mathbb{V}^\star$.  We accomplish this by using the following subroutine

\begin{algorithm}
	\KwData{The final values for the elements of $\mathbb{V}$, $\bigl\{ \mathbb{V}_i, \mathbb{M}_i \bigr\}$, and the number of columns in $\mat{L}$, $n$.}
	\KwResult{The sets of $\mathbb{V}^\star$, with each set in $\mathbb{V}^\star$ representing a column cluster and containing its elements. }
	\Begin{
	\textbf{int} nClust = 1; $\mathbb{V}^\star \left[ nClust \right] = \mathbb{V} \left[ 1, 1 \right]$ \;
	\For { $i = 2:n$ } {
		\textbf{boolean} isNew = \textbf{true}; $\mathbb{V}_i = \mathbb{V} \left[ i, 1 \right]$ \;
		\For { $j = 1: \textrm{nClust}$ } {
			$\mathbb{V}^\star_j = \mathbb{V}^\star \left[ j \right]$ \;
			\If { $\mathbb{V}^\star_j = \mathbb{V}_i$ } {
				isNew = \textbf{false} \;
				\textbf{break} \;
			} 
		}
		
		\If { isNew } {
			nClust ++ \;
			$\mathbb{V}^\star \left[ nClust \right] = \mathbb{V}_i$ \;
		}
	} 
}
\caption{Compute the set of all column clusters sets, $\mathbb{V}^\star$. }
\end{algorithm}















































\end{large}
\end{flushleft}
\end{document}