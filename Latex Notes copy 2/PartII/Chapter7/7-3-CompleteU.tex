\documentclass[../ClusteringConnectionsMAIN.tex]{subfiles}
 



\begin{document}
\begin{flushleft}
\begin{large}


The elements of $\mathbb{U}$ must now be "\emph{completed}" so that they include any indirectly related elements.  Taking advantage of the disjointedness of $\mat{L}$, the "\emph{completion}" of the elements in $\mathbb{U}$ can be accomplished using the simple subroutine below

\begin{algorithm}
	\KwData{Initial values for the elements of $\mathbb{U}$, $\bigl\{ \mathbb{U}_i, \mathbb{M}_i \bigr\}$, and the number of rows in $\mat{L}$, $m$.}
	\KwResult{The final values for each of the $\bigl\{ \mathbb{U}_i, \mathbb{M}_i \bigr\} \in \mathbb{U}$. }
	\Begin{
	\textbf{boolean} isChanged = \textbf{true} \;
	\While {isChanged} {
		isChanged = \textbf{false} \;
		\For { $i = 1:m$ } {
			$\mathbb{U}_i = \mathbb{U} \left[ i, 1 \right]$; $\mathbb{M}_i = \mathbb{U} \left[ i, 2 \right]$ \;
			\For { $j = 1:m$ } {
				$\mathbb{U}_j = \mathbb{U} \left[ j, 1 \right]$; $\mathbb{M}_j = \mathbb{U} \left[ j, 2 \right]$ \;
				\If { $\mathbb{U}_i \neq \mathbb{U}_j$ \&\& $\mathbb{U}_i \bigcap \mathbb{U}_j \neq \emptyset$ } {
					$\mathbb{U}_i = \mathbb{U}_i \bigcup \left\{ \mathbb{U}_j \ \mathbb{U}_i \right\}$ \;
					$\mathbb{M}_i = \mathbb{M}_i \bigcup \left\{ \mathbb{M}_j \ \mathbb{M}_i \right\}$ \;
					$\mathbb{U}_j = \mathbb{U}_j \bigcup \left\{ \mathbb{U}_i \ \mathbb{U}_j \right\}$ \;
					$\mathbb{M}_j = \mathbb{M}_j \bigcup \left\{ \mathbb{M}_i \ \mathbb{M}_j \right\}$ \;
					isChanged = \textbf{true} \;
				} 
				\If { isChanged } {
					$\mathbb{U} \left[ j, 1 \right] = \mathbb{U}_j$; $\mathbb{U} \left[ j, 2 \right] = \mathbb{M}_j$ \;
				}
			}
			\If { isChanged } {
				$\mathbb{U} \left[ i, 1 \right] = \mathbb{U}_i$; $\mathbb{U} \left[ i, 2 \right] = \mathbb{M}_i$ \;
			} 
		} 
	}
}
\caption{Computing the final value for each $\bigl\{ \mathbb{U}_i, \mathbb{M}_i \bigr\} \in \mathbb{U}$.}
\end{algorithm}


\end{large}
\end{flushleft}
\end{document}