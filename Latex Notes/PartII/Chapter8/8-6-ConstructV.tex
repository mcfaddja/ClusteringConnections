\documentclass[../../ClusteringConnectionsMAIN.tex]{subfiles}
 



\begin{document}
\begin{flushleft}
\begin{large}


We initially construct $\mathbb{V}$ by following the sub-routine (sub algorithm) given below

\begin{algorithm}
	\KwData{Connection Matrix for columns, $\mat{A}_V$, and the number of columns in $\mat{L}$, $n$.  }
	\KwResult{Initial value for each of the $\bigl\{ \mathbb{V}_j, \mathbb{N}_j \bigr\} \in \mathbb{V}$. }
	\Begin{
	\For { $j = 1:n$ } {
		$\mathbb{V}_j = \emptyset$; $\mathbb{N}_j = \emptyset$ \;
		\For { $i = 1:n$ } {
			\If { $\left( \mat{A}_V \right)_{ji} \neq 0$ } {
				$\mathbb{V}_j = \mathbb{V}_j \bigcup \left\{ i \right\}$ \;
				$\mathbb{N}_j = \mathbb{N}_j \bigcup \left\{ \left( \mat{A}_V \right)_{ji} \right\}$ \;
			} 
		}
		$\mathbb{V} \left[ j, 1 \right] = \mathbb{V}_j$; $\mathbb{V} \left[ j, 2 \right] = \mathbb{N}_j$\; 
	} 
}
\caption{Computing the initial value for each of the $\bigl\{ \mathbb{V}_j, \mathbb{N}_j \bigr\} \in \mathbb{V}$.}
\end{algorithm}

After the completion of this sub-routine, we will make a copy of this initial state of $\mathbb{V}$.  We denote this copy of this initial state as $\mathbb{V}_0$.








































\end{large}
\end{flushleft}
\end{document}