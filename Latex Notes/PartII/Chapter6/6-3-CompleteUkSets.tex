\documentclass[../../ClusteringConnectionsMAIN.tex]{subfiles}
 



\begin{document}
\begin{flushleft}
\begin{large}

The elements of $\mathbb{U}_k^\text{ (rel)}$ should contain the indices of all rows in the same cluster as each element.  That is to say, each $\left( \mathbb{U}_k^\text{ (rel)} \right)_i$ should be the set containing the indices of all the rows in the same cluster as row $i$, in addition to the index of row $i$; however, this is not guaranteed to be the case initially after the creation of the $\left( \mathbb{U}_k^\text{ (rel)} \right)_i$ as described above.  To illustrate this, consider the values for some $\mat{U}_k^\text{ (conn)}$ as given below

\begin{align}
\mat{U}_k^\text{ (conn)} = \left[
\begin{array}{c c c c c c}
	1 & 1 & 0 & 0 & 0 & 0 \\
	0 & 1 & 0 & 0 & 0 & 0 \\
	0 & 1 & 1 & 0 & 0 & 0 \\
	0 & 0 & 0 & 1 & 0 & 0 \\
	0 & 0 & 0 & 0 & 1 & 0 \\
	0 & 0 & 0 & 0 & 1 & 1
\end{array} \right]
\end{align}

Using the definition for the $\left( \mathbb{U}_k^\text{ (rel)} \right)_i$ from (1.8), we obtain the following initial values for the $\left( \mathbb{U}_k^\text{ (rel)} \right)_i$ generated from the $\mat{U}_k^\text{ (conn)}$ expressed above

\begin{center}
\begin{tabular}{c | l}
$i$ & $\mathbb{U}_k^\text{ (rel)}$ \\
\hline
$1$ & $\left( \mathbb{U}_k^\text{ (rel)} \right)_1 = \left\{ 1, 2 \right\}$ \\
$2$ & $\left( \mathbb{U}_k^\text{ (rel)} \right)_2 = \left\{ 2 \right\}$ \\
$3$ & $\left( \mathbb{U}_k^\text{ (rel)} \right)_3 = \left\{ 2, 3 \right\}$ \\
$4$ & $\left( \mathbb{U}_k^\text{ (rel)} \right)_4 = \left\{ 4 \right\}$ \\
$5$ & $\left( \mathbb{U}_k^\text{ (rel)} \right)_5 = \left\{ 5 \right\}$ \\
$6$ & $\left( \mathbb{U}_k^\text{ (rel)} \right)_6 = \left\{ 5, 6 \right\}$ 
\end{tabular}
\end{center}
\begin{center}
\emph{\textbf{Table ??:} Example of initial values of $\mathbb{U}_k^\text{ (rel)}$}
\end{center}

Clearly, rows 1, 2, and 3 belong in the same cluster; rows 5 and 6 also belong to the same cluster, but one different than the first; and row 4 belongs by itself.  However, with the exception of rows 4 and 6, the $\left( \mathbb{U}_k^\text{ (rel)} \right)_i$ do not reflect this, at least not initially.  Thus, each of the $\left( \mathbb{U}_k^\text{ (rel)} \right)_i$ must be completed by adding the other indices of rows which are in the same cluster as the row with which $\left( \mathbb{U}_k^\text{ (rel)} \right)_i$ is associated.  This will be accomplished using the intersect, $\left( \mathbb{U}_k^\text{ (rel)} \right)_i \bigcap \left( \mathbb{U}_k^\text{ (rel)} \right)_j$, and union, $\left( \mathbb{U}_k^\text{ (rel)} \right)_i \bigcup \left( \mathbb{U}_k^\text{ (rel)} \right)_j$, set operations.  \newline

We will use the intersect set operation will indicate if the rows represented by $\left( \mathbb{U}_k^\text{ (rel)} \right)_i$ and $\left( \mathbb{U}_k^\text{ (rel)} \right)_j$ belong to the same cluster, as the operation will give

\begin{align*}
\left( \mathbb{U}_k^\text{ (rel)} \right)_i \bigcap \left( \mathbb{U}_k^\text{ (rel)} \right)_j = \emptyset
\end{align*}

if rows $i$ and $j$ belong to different clusters.  If this intersection yields a non-empty set, then we will use the second set operation, the union 

\begin{align*}
\left( \mathbb{U}_k^\text{ (rel)} \right)_i \bigcup \left( \mathbb{U}_k^\text{ (rel)} \right)_j
\end{align*}

to combine the elements of $\left( \mathbb{U}_k^\text{ (rel)} \right)_i$ and $\left( \mathbb{U}_k^\text{ (rel)} \right)_j$.  This discussion allows us to layout the following process for completing all of the $\left( \mathbb{U}_k^\text{ (rel)} \right)_i \in \mathbb{U}_k^\text{ (rel)}$.  

\begin{algorithm}
	\KwData{Initial values for the $\left( \mathbb{U}_k^\text{ (rel)} \right)_i \in \mathbb{U}_k^\text{ (rel)}$}
	\KwResult{Completed values for the $\left( \mathbb{U}_k^\text{ (rel)} \right)_i \in \mathbb{U}_k^\text{ (rel)}$, fully representing the cluster of each row by containing all elements of that cluster.}
%\textbf{Begin} \;
\Begin{
	\For { $i = 1:m$ } {
		\For { $j = 1:m$ } {
			\If { $\left( \mathbb{U}_k^\text{ (rel)} \right)_i \bigcap \left( \mathbb{U}_k^\text{ (rel)} \right)_j \neq \emptyset$ } {
				$\left( \mathbb{U}_k^\text{ (rel)} \right)_i = \left( \mathbb{U}_k^\text{ (rel)} \right)_i \bigcup \left( \mathbb{U}_k^\text{ (rel)} \right)_j$ \;
				$\left( \mathbb{U}_k^\text{ (rel)} \right)_j = \left( \mathbb{U}_k^\text{ (rel)} \right)_j \bigcup \left( \mathbb{U}_k^\text{ (rel)} \right)_i$ \;
			} 
		}
	} 
}
\caption{Completing the sets $\left( \mathbb{U}_k^\text{ (rel)} \right)_i$, $\forall i \in \left[ i, m \right]$}
\end{algorithm}

Applying this algorithm to the example in \emph{\textbf{Table ??}}, we have 


\begin{center}
\begin{tabular}{c | l ||| l | l || l || l}
$i$ & $\mathbb{U}_k^\text{ (rel)}$ & $i = 1$ & $i = 1$ & $i = 2$ & $i = 5$  \\
 & & $j = 2$ & $j = 3$ & $j = 2$ & $j = 6$ \\
\hline
\hline
$1$ & $\left( \mathbb{U}_k^\text{ (rel)} \right)_1 = \left\{ 1, 2 \right\}$ & & $=\left\{ 1, 2, 3 \right\}$ & &  \\
\hline
$2$ & $\left( \mathbb{U}_k^\text{ (rel)} \right)_2 = \left\{ 2 \right\}$ & $=\left\{ 1, 2 \right\}$ & & $=\left\{ 1, 2, 3 \right\}$ &  \\
\hline
$3$ & $\left( \mathbb{U}_k^\text{ (rel)} \right)_3 = \left\{ 2, 3 \right\}$ &  & $=\left\{ 1, 2, 3 \right\}$ & &  \\
\hline
$4$ & $\left( \mathbb{U}_k^\text{ (rel)} \right)_4 = \left\{ 4 \right\}$ & & & &  \\ 
\hline
$5$ & $\left( \mathbb{U}_k^\text{ (rel)} \right)_5 = \left\{ 5 \right\}$ & & & & $=\left\{ 5, 6 \right\}$  \\
\hline
$6$ & $\left( \mathbb{U}_k^\text{ (rel)} \right)_6 = \left\{ 5, 6 \right\}$ & & & &
\end{tabular}
\end{center}
\begin{center}
\emph{\textbf{Table ??:} Example of algorithm on the elements of $\mathbb{U}_k^\text{ (rel)}$,} \\
\emph{with steps and elements without change omitted.}
\end{center}



\end{large}
\end{flushleft}
\end{document}