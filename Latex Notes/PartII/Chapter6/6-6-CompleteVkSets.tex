\documentclass[../../ClusteringConnectionsMAIN.tex]{subfiles}
 



\begin{document}
\begin{flushleft}
\begin{large}

The elements of $\mathbb{V}_k^\text{ (rel)}$ should contain the indices of all columns in the same cluster as each element.  That is to say, each $\left( \mathbb{V}_k^\text{ (rel)} \right)_i$ should be the set containing the indices of all the columns in the same cluster as column $i$, in addition to the index of column $i$; however, this is not guaranteed to be the case initially after the creation of the $\left( \mathbb{V}_k^\text{ (rel)} \right)_i$ as described above. The same illustration we gave for the $\left( \mathbb{U}_k^\text{ (rel)} \right)_i$ \textbf{Section 1.2.2} illustrates this point as well.  Similar to the case with the rows, each of the $\left( \mathbb{V}_k^\text{ (rel)} \right)_i$ must be completed by adding the other indices of columns which are in the same cluster as the column with which $\left( \mathbb{V}_k^\text{ (rel)} \right)_i$ is associated.  This will be accomplished using the intersection, $\left( \mathbb{V}_k^\text{ (rel)} \right)_i \bigcap \left( \mathbb{V}_k^\text{ (rel)} \right)_j$, and union, $\left( \mathbb{V}_k^\text{ (rel)} \right)_i \bigcup \left( \mathbb{V}_k^\text{ (rel)} \right)_j$, set operations.  \newline

We will use the intersect set operation will indicate if the columns represented by $\left( \mathbb{V}_k^\text{ (rel)} \right)_i$ and $\left( \mathbb{V}_k^\text{ (rel)} \right)_j$ belong to the same cluster, as the operation will give

\begin{align*}
\left( \mathbb{V}_k^\text{ (rel)} \right)_i \bigcap \left( \mathbb{V}_k^\text{ (rel)} \right)_j = \emptyset
\end{align*}

if columns $i$ and $j$ belong to different clusters.  If this intersection yields a non-empty set, then we will use the second set operation, the union 

\begin{align*}
\left( \mathbb{V}_k^\text{ (rel)} \right)_i \bigcup \left( \mathbb{V}_k^\text{ (rel)} \right)_j
\end{align*}

to combine the elements of $\left( \mathbb{V}_k^\text{ (rel)} \right)_i$ and $\left( \mathbb{V}_k^\text{ (rel)} \right)_j$.  This discussion allows us to layout the following process for completing all of the $\left( \mathbb{V}_k^\text{ (rel)} \right)_i \in \mathbb{V}_k^\text{ (rel)}$.  

\begin{algorithm}
	\KwData{Initial values for the $\left( \mathbb{V}_k^\text{ (rel)} \right)_i \in \mathbb{V}_k^\text{ (rel)}$}
	\KwResult{Completed values for the $\left( \mathbb{V}_k^\text{ (rel)} \right)_i \in \mathbb{V}_k^\text{ (rel)}$, fully representing the cluster of each column by containing all elements of that cluster.}
\Begin{
	\For { $i = 1:n$ } {
		\For { $j = 1:n$ } {
			\If { $\left( \mathbb{V}_k^\text{ (rel)} \right)_i \bigcap \left( \mathbb{V}_k^\text{ (rel)} \right)_j \neq \emptyset$ } {
				$\left( \mathbb{V}_k^\text{ (rel)} \right)_i = \left( \mathbb{V}_k^\text{ (rel)} \right)_i \bigcup \left( \mathbb{V}_k^\text{ (rel)} \right)_j$ \;
				$\left( \mathbb{V}_k^\text{ (rel)} \right)_j = \left( \mathbb{V}_k^\text{ (rel)} \right)_j \bigcup \left( \mathbb{V}_k^\text{ (rel)} \right)_i$ \;
			} 
		}
	} 
}
\caption{Completing the sets $\left( \mathbb{V}_k^\text{ (rel)} \right)_i$, $\forall i \in \left[ i, n \right]$}
\end{algorithm}



\end{large}
\end{flushleft}
\end{document}