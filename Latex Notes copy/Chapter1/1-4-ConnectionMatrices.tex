\documentclass[../ClusteringConnectionsMAIN.tex]{subfiles}
 



\begin{document}
\begin{flushleft}
\begin{large}


We will create two "\emph{Connection Matrices}" from $\mat{L}$.  These matrices represent the connections among all of the row elements of $\mat{L}$ or among all of the column elements of $\mat{L}$.  They will provide the information required for clustering either the rows of $\mat{L}$ or for clustering the columns of $\mat{L}$.


\subsection{Row Connection Matrix}

The first of the "\emph{Connection Matrices}" that we will create from $\mat{L}$ is the "\emph{Row Connection Matrix}".  This matrix will be used to provide information for clustering the rows of $\mat{L}$.  We will denote this matrix by $\mat{A}_U \in \R^{m \times m}$ and define it, in terms of $\mat{L}$, according to the expression

\begin{align}
\mat{A}_U = \mat{L} \, \mat{L}^\intercal
\end{align}

We may also use the alternate definition 

\begin{align}
\left( \mat{A}_U \right)_{ij} = \sum_{k = 1}^n \left\{ \delta^\star \left( \mat{L}, i, k \right) \, \mat{L}_{jk} \right\} \tag{1.4.1a}
\end{align}

where $i, j \in \left[ 1, m \right] \subset \Z^+$.


\subsection{Column Connection Matrix}

The second "\emph{Connection Matrix}" to be created from $\mat{L}$ is the "\emph{Column Connection Matrix}".  This matrix will be used to provide information for clustering the columns of $\mat{L}$.  We will denote this matrix by $\mat{A}_V \in \R^{n \times n}$ and define it, in terms of $\mat{L}$, according to the expression

\begin{align}
\mat{A}_V = \mat{L}^\intercal \, \mat{L}
\end{align}

We may also use the alternate definition 

\begin{align}
\left( \mat{A}_V \right)_{ij} = \sum_{k = 1}^m \left\{ \delta^\star \left( \mat{L}, k, i \right) \, \mat{L}_{kj} \right\} \tag{1.4.2a}
\end{align}

where $i, j \in \left[ 1, n \right] \subset \Z^+$.


\end{large}
\end{flushleft}
\end{document}